% !Mode:: "Tex:UTF-8"
\documentclass{xcumcmart}
\usepackage{setspace}
\usepackage{enumerate}
\usepackage{graphics}
\usepackage{tabu}

% \title{text}这里是显示在第三页的文章标题
\title{机器学习引论作业二}
\author{何长鸿 2016141482154}

\linespread{1.2} %行距
% \setlength{\parskip}{1.2em} %1.4倍段落距离

\begin{document}
\renewcommand\arraystretch{2}
\maketitle
\section{Question1}
\textbf{Q:}Who is Vladimir N. Vapnik?\\
\par \textbf{A:}He is one of the main developers of the Vapnik–Chervonenkis theory of statistical learning,[1] and the co-inventor of the support-vector machine method, and support-vector clustering algorithm.

\section{Question2}
\textbf{Q:}How to compute the distance between a given data point to the boundary?\\
\par \textbf{A:}For the i-th point $x_i$ in data, we caculate the distance from point to the hyperplane with the expression $$r=\frac{W^Tx_i+b}{||W||}$$ where $W$ is weights of the hyperplane, $b$ is the constant term.
\end{document}